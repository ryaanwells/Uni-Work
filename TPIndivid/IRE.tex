\documentclass{report}
\usepackage{verbatim}

\title{Individual Writing Exercise}
\author{Ryan Wells -- 1002253w}
\begin{document}

\maketitle
\section*{Introduction}

\emph{The problem is not as easy as many sports writers would have you believe... The problem is not as difficult as many mathematicians would have you believe.}

A New Property And A Faster Algorithm For Baseball Elimination\\
Kevin D. Wayne\\

In the competitive world of professional sport, speculation over who or what team will finish at the top of a league occurs frequently and often with arbitrary decisions based on preference and intended audience. Some sports writers have pushed this speculation to claim certain teams or individuals have absolutely no chance of finishing top of their league based on their mathematical calculations. As interesting, and sometimes controversial as this may be, na\"{\i}ve calculations posed by sports pundits (such as one team or individual being eliminated from success as the total of their current wins and games still to play is less than the current league leader) are often easily calculated or obvious and underlying more interesting results may be found through an in-depth analysis of the league. This project aims to fulfil this analysis with a retrospective view on American Baseball Leagues from the 2011-2012 season. A working knowledge of graph theory is helpful, however the specific elements utilised in this project are outlined in detail. 

The ability to prove the elimination of a team or individual, proving that there is no possible outcome where this team or individual can finish top of a league, is a relatively easy curiosity to sate thanks to the work done by B. L. Schwarts proving that a maximum flow computation can determine whether a single team or individual is eliminated from a league. 

A graph is constructed involving every team or individual and every match remaining in the league that does not involve the team for which we are checking the elimination status of. From the source, an ark is created to each match node with a weight equal to the number of times that match is to be played, two arks are created from this match node to its respective team nodes with an infinite weight, and finally one ark from each team node to the sink with a weight of the maximum score obtainable by the team we are checking the elimination status of (wins + games remaining) minus the number of wins of the team that corresponds to the team node we are coming from. 

By computing a maximum flow computation on this graph we can determine the elimination status of a specific team. If a maximum flow is a saturating flow, that is to say no more flow can leave the source node, then a team is not eliminated; and if a maximum flow is not saturating then a team is eliminated.

Wayne proves a refinement on this idea where, if all teams or individuals in a league are ranked in a non-descending order by their wins and games remaining then if one team is eliminated then so are all the teams or individuals with the same or lower rank. This is a useful refinement that is utilized in our project. 

\break
Complex computations such as these show their true worth in real life contexts. The information obtainable by these processes are of interest to not only professionals in the sport industry but also to fans and avid followers of teams or individuals to bookmakers and sports writers, and as such we intend to make this information readily available to anyone who wishes to view this. 

We have chosen to use the American Baseball Leagues for our project due to the ease of access to accurate data and its scoring system. We are currently only concerned with the main leagues and not playoffs or wildcard games.

We intend to have a desktop or web based application in which you can view a current or previous league and relevant information about this league with real time updating, where appropriate. It is hoped that through displaying this information our audience can be better informed and take an interest in the more complex methods of deduction behind our calculations.

The project intends to create three main modules: a functional and aesthetically pleasing user interface where the user can view information about teams or individuals in a certain league, a parser for a web page or RSS feed and a parser for text files so the user can get the most up to date information about their league but also view standings retrospectively, and an implementation of an algorithm that utilizes the above theorems to accurately determine a teams elimination status. A user should be able to navigate forwards and backwards through a league by week, by game and by next elimination event. Each of these navigation milestones have their own benefits and it is agreed that all have merit to be in this project.

An expansion to this project could be to allow a user to track their own league. The user would input the standings and teams into the program and update the scores as they came in, allowing the user to see a more localised real life example of this project. A user would also be able to create a league for their own learning where they could input results to provoke an elimination to aid in understanding how the process works. This project is not designed to be a learning tool, but a simple addition such as this can make it function as one. This would be an interesting addition to the program from the end user point of view and is currently under consideration. 

The specific implementation of each module is established in the following sections of this report along with empirical evaluations and our process in developing and refinement of each module with possible improvements on each.
\end{document}
